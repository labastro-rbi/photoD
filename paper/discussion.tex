\section{Discussion and Conclusions \label{sec:disc}}


Distances to stars are a crucial ingredient in our quest to better understand the formation and evolution
of the Milky Way galaxy. Anticipating photometric catalogs with tens of billions of stars from Rubin's LSST,
here we presented PhotoD, a pipeline for computing Bayesian distance estimates that can handle LSST-sized datasets. 

The Bayesian procedure implemented in the PhotoD pipeline builds on previous work \citep[e.g.,][]{2011MNRAS.411..435B, 2012ApJ...757..166B,
2014ApJ...783..114G, green_3d_2019, bailer-jones_estimating_2021} and improves it in several important ways:
i) the use of multiple stellar populations (in addition to the dominant main-sequence stars)  
ii) improved color tracks for main-sequence stars and (especially) red giants, including the use of very young ($<$1 Gyr) populations 
               and an extended $[Fe/H]$ range, and
iii) priors based on sophisticated TRILEGAL simulations \citep{2022ApJS..262...22D} that include multiple stellar populations and
       account for all principal structural components of the Galaxy.

Extensive, although still preliminary, testing demonstrates the expected statistical behavior of implemented computations and
that SDSS-based luminosity-color sequences are supported by more recent direct (trigonometric) distance measurements by Gaia.
Tests of pipeline performance show that a sample of 10$^{10}$ stars can be processed in a few days using a moderate-size cluster.
We intend to process each LSST Data Release as they become available and make outputs accessible via Rubin Science Platform. 

We anticipate that the accuracy of resulting distance estimates, as well as estimates of metallicity and interstellar dust extinction
along the line of sight, will improve as LSST advances because:
\begin{itemize}
\item As more LSST data are collected, not only that photometric depth will improve but photometric calibration will improve as well.
\item Photometric light curves will improve identification of variable populations such as quasars, RR Lyrae stars and eclipsing
           binary stars. When proper motion measurements become available, separation of nearby stars from distant stars will improve, too. 
\item Luminosity-color sequences will be recalibrated in LSST's photometric system and improved using LSST's own globular cluster data and Gaia's parallax measurements.
\item The extension of photometric coverage to longer IR wavelengths (e.g., using Euclid and Roman Space Telescope survey photometry)
          will improve constraints on model parameters, especially in Galactic plane regions with large dust extinction (for more details,
          see \citealt{2012ApJ...757..166B}).
\item Improvements in our understanding of the Milky Way structure will iteratively lead to improvements of Galaxy models
         such as TRILEGAL, and in turn to improvements of Bayesian priors.
\item In the context of interstellar dust studies, it may be worthwhile to consider hiearchical Bayesian modeling (e.g., setting priors
             for the line-of-sight reddening profile, as discussed by \citealt{2014ApJ...783..114G}). 
\end{itemize}

These improvements will require substantial additional work, but given the implied transformative impact of resulting
distance estimates on our understanding  of the formation and evolution of the Milky Way, it seems well justified.  