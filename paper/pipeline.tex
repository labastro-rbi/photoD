

\section{All-sky PhotoD Pipeline Implementation \label{sec:LSDBpipe}}


% {\bf Lovro:} will here describe LSDB implementation (with Neven's help) and pipeline performance on the full TRILEGAL catalog...

Implementation of the Bayesian algorithm described in preceding sections assumes that priors can be assumed constant over some sky region, hereafter a patch (see \S\ref{sec:priors}). In practice,  patches need to be at least several sq. deg. large in order to have enough stars to compute priors. At the same time, they should not be too large to satisfy the assumption of constancy. Both of these size limits vary across the sky due to Galactic structure effects. 

The scale of the dataset expected from LSST -- O(10Bn) objects -- makes this
computation challenging to perform on a single core or computer. Ideally,
one would wish to execute the computation in parallel on many (hundreds) of cores,
likely housed on a number of separate nodes. This requires a distributed
processing framework capable of orchestrating such a solution.

\subsection{LSDB and Healpix Implementation} 

For our work, we've adopted the \href{https://lsdb.readthedocs.io/}{\em Large Survey Database} (LSDB; \cite{2024AAS...24326109W})
framework.  LSDB is a distributed computing framework written in Python,
which enables multi-node and multi-core computation on large (PB-scale),
partitioned, catalog datasets.  It extends the concepts of the LSD toolkit
\citep{2011AAS...21743319J}
to astronomical catalogs where the distribution of objects on the sky is
very uneven -- such as is the case for Galactic star counts.  Leveraging
broadly adopted community libraries such as astropy, Pandas, and Dask, LSDB
presents a user-friendly API on which it is possible to build our pipeline.

Specifically, LSDB allows us to define a piece of computation (the PhotoD
function, written in Python) that should be executed either on every object in
the catalog, or on a per-patch basis (where the patch has a potentially
dynamically sized solid angle). For example, we compute priors for each of the 
patches (\S\ref{sec:priors}). LSDB then ensures that these functions are successfully run over the entire (arbitrarily sized) catalog. LSDB handles the data distribution and
movement between multiple nodes in the cluster, automatically handles
transient failures, and writes the output (in parallel) to another
partitioned, searchable, table.

% Using this approach, we were able to process simulated catalogs up to XX in
% size, using YY nodes, in XX wall-clock hours. Extrapolating to expectations
% for Rubin's DR1, we calculate the same cluster will be able to repeat this
% computation on real data in ZZ hours.


%One of the recent tools developed to efficiently loop across the sky patches and perform some per-patch calculations is the \href{https://lsdb.readthedocs.io/}{Large Survey Database}\footnote{https://lsdb.readthedocs.io/}.

The combination of the HiPSCat and LSDB workflow are particularly efficient and useful for a science case tackled by the PhotoD. HiPSCat effectively partitions the sky in pixels based on the density of the sources, therefore creating pixels that contain roughly the same number of stars. This strategy is appropriate both scientifically, as the resolution of the map (area subtended by the pixel) depends on the number of stars within (e.g. smaller pixels and higher resolution in the galactic plane and lower in the halo), and computationally because the size of the each chunk that is sent to processing is approximately the same, which therefore optimizes the execution of the code. In our particular case, we are using HiPSCat pixels that are a few hundred megabytes in size.
 
 
\subsection{Pipeline Performance}

We have performed several benchmarks to estimate the performance of the PhotoD code. In our initial tests based on the SDSS Stripe 82 data we found that the distances for a billion stars can be estimated in about 100 days on a Mac M1 Pro laptop using one core (10 milliseconds per star). 

In further testing, we used outputs from the TRILEGAL simulation and found that the parallel processing performed with LSDB scaled well with respect to the single core values. On a local cluster node with dual AMD EPYC 9474F 48-core processors we were able to achieve per-star figures similar to single-core processing, therefore reducing the required processing time to about a week. Our ongoing tests are showing that when all 10 nodes (400 cores) of the Ruđer Bošković Institute's Narval cluster are used, we may be able to estimate distances for about 10 billion stars in about two days. 

The output catalog containing LSST photometry (with errors) and minimalistic auxiliary metadata for 10 billion stars are would amount to about 1 TB of data. The code outputs, including a covariance matrix of the three fitted parameters, would have approximately the same volume. 
