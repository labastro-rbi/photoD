\section{Introduction} \label{sec:intro}


In order to map the Milky Way in three dimensions, distances to its stars must be accurately  estimated.
Kinematic studies based on proper motion data also require estimates of stellar distances.  There are a variety
of astronomical methods to estimate distances to stars, ranging from direct geometric (trigonometric) methods
for nearby stars to indirect methods based on astrophysics for more distant stars.

As demonstrated with SDSS \citep{2008ApJ...673..864J, 2008ApJ...684..287I}, Pan-STARRS \citep{2014ApJ...783..114G, green_3d_2019} and most recently with
Gaia data \citep{bailer-jones_estimating_2021}, broad-band near-UV to near-IR stellar photometry is sufficient
to estimate distance, metallicity and interstellar dust extinction along the line of sight for stars in the Galaxy.
In analogy with photo-z, photometric redshifts of galaxies, hereafter we refer to these methods as photo-D. 
The photo-D method is conceptually quite simple: multi-dimensional color tracks (either empirical or model-based),
parametrized by luminosity, metallicity and extinction, are fit to observed colors and the best-fit produces estimates
of the three model parameters. The method relies on strong correlations between stellar colors and
stellar luminosity for dominant stellar populations such as main-sequence stars, red giants, white dwarfs, and even
for the majority of unresolved binary stars. These are the same correlations that are responsible for the abundant
structure seen in the Hertzsprung-Russell diagram. In addition to colors and luminosity, these correlations
also involve metallicity in case of main-sequence and red giant stars, and surface gravity in case of white dwarfs.
For the youngest main-sequence stars, stellar age may play a role, too. Measured stellar colors are also affected
by interstellar dust extinction along the line of sight towards the star. Consequently, sufficiently accurate measurements of apparent
brightness and sufficient number of UV to IR colors, such as those that Rubin Observatory's Legacy Survey of Space
and Time (LSST; \citealt{2019ApJ...873..111I}) will provide, can be used to accurately estimate these parameters,
and ultimately stellar distances.

LSST-based stellar distance estimates will significantly improve available distance catalogs, such as those recently produced by \cite{green_3d_2019}
and \cite{bailer-jones_estimating_2021}. First, the sample size will be increased by more than an order of magnitude, and exceed 10 billion stars.
Distance accuracy for stars with sufficiently small photometric errors will be within the 5-10\% range, or about twice
as accurate as for surveys lacking the UV $u$ band (which provides metallicity constraints). LSST-based stellar distances
will reach about 10 times further than Gaia’s color-based distances and will be transformative for studies of the
Milky Way in general, and of its stellar and dark matter halo in particular. 

In this paper we present a Bayesian procedure and pipeline that build on previous work and can handle LSST-sized datasets.
In \S2, we describe methodology and in \S3 we test the pipeline using both simulated catalogs, and SDSS, DECam and Gaia
photometry. We discuss possibilities for further improvements and catalog public release plans in \S4. 
