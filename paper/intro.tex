\section{Introduction} \label{sec:intro}

Thanks to the \textit{Vera C. Rubin} observatory's \textit{Legacy Survey of Space and Time} (LSST), for the first time in history, an astronomical catalog will contain more Milky Way stars than there are living people on Earth -- of the order 10-20 billion, depending on model assumptions. In order to map the Milky Way in three dimensions, distances to these stars must be accurately  estimated. In this paper we describe a method that will deliver LSST-based stellar distance estimations complementary to \textit{Gaia's} state-of-the-art trigonometric parallaxes and reach about 10 times further, to approximately 100 kpc. These results will be transformative for the studies of the Milky Way in general, and the stellar and the dark matter halo in particular as never before was there a survey that simultaneously observed roughly two thirds of the sky, to the co-added depth of \textit{r}$\approx$26 mag. 

\magcom{A bit about the importance of the distance estimation in the MW, dust implications (for extragalactic science too).}

There are a variety of astronomical methods to estimate distances to stars, ranging from direct geometric (trigonometric) methods for nearby stars to indirect methods based on astrophysics for more distant stars. 

\magcom{Mention \cite{bailer-jones_estimating_2021}, \cite{gordon_panchromatic_2016}, \cite{green_measuring_2014,green_three-dimensional_2015,green_3d_2019}, \cite{juric_milky_2008} and \cite{lallement_3d_2014}, \cite{queiroz_starhorse_2018}}.

Layout of the paper is...
