

\section{Results\label{sec:results}}

Here we test the validity of the Bond et al. kinematic models for disk and halo stars, as enabled by the high-metallicity FGKM sample
and the red giant sample introduced in the preceding Section.
We first discuss the behavior of the first two statistical moments (mean and dispersion) for 3-dimensional velocity distribution
and then compare the predicted and observed shapes of the rotational velocity distribution in more detail. We test
model predictions for the behavior of quantities directly measured by Gaia, stellar radial velocities and proper motions, across
the whole sky. Finally, we verify that the behavior of Gaia data for halo stars is consistent with a velocity ellipsoid that is
spatially invariant when expressed in spherical coordinates. 


\subsection{Comparison of predicted and observed 3-dimensional velocity distribution for disk stars}

We first emulate the top left panels in Figures 5, 7 and 11 from the Bond et al. study. We select stars towards the North and South
Galactic poles ($|b|>80^\circ$) and show the variation of their velocity components with $|Z|$ in \autoref{fig:3v}. Due to the
galactic latitude constraint, the $v_Z$ component is dominated by the radial velocity measurements while the other two components
are dominated by proper motion measurements. Compared to the Bond et al. high-luminosity sample, the distance range probed
by Gaia's FGKM sample is closer: from about 50 pc to about 1 kpc vs. 0.8--5 kpc (at the far end, the SDSS sample is limited by
sample contamination due to halo stars).

The Bond et al. model predictions for the velocity mean and dispersion are in agreement with Gaia's observations. In particular, the
gradient of the rotational velocity with $|Z|$ is evident, and the extrapolation of SDSS rotational velocity measurements to $Z=0$
is quantitatively supported by Gaia. The same conclusions remain valid when the full sample is considered (that is, without the $|b|>80^\circ$ restriction).

The $<$1 kpc distance range probed by FGKM stars is significantly extended with the red giant sample, as illustrated in
\autoref{fig:3vRGs}. In conclusion, the Bond et al. models for the velocity mean and dispersion of disk stars appear validated
in the $|Z|$ range 0--5 kpc, without any appreciable north-south asymmetry.  


\begin{figure}[!ht]
\includegraphics[width=0.999\textwidth,angle=0]{figures/plot3VvsZ_NGP.png}
\includegraphics[width=0.999\textwidth,angle=0]{figures/plot3VvsZ_SGP.png}
\caption{The variation of cylindrical velocity components with distance from the plane, $|Z|$, for FGKM stars with measured radial velocities, shown as dots. 
The top row shows 6,548 stars towards the North Galactic Pole ($b >80^\circ$) and the bottom row shows 7,578 stars towards the South Galactic Pole
($b < -80^\circ$). Triangles show the mean values in bins of $|Z|$ and the thick dashed lines show the $\pm2\sigma$ envelope around means, where
$\sigma$ is the standard deviation for each bin (i.e., velocity dispersion). The thick dot-dashed lines are models for the mean velocity (0 for $v_R$ and
$v_Z$, and given by eq.~\ref{eq:Dlag} for $v_\phi$). The thin dot-dashed lines show the $\pm2\sigma$ envelope, with $\sigma$ (velocity dispersion)
given by eqs.~\ref{eq:Ddisp}, \ref{eq:DdispR}, and \ref{DdispZ}, for $\phi$, $R$ and $Z$ panels, respectively.  The thin dashed lines in the two left panels
also show $\pm2\sigma$ envelopes, but with the velocity dispersion computed using eq.~\ref{eq:pDvPhi} (see \S~\ref{sec:vshape}).} 
\label{fig:3v} 
\end{figure}

 
\begin{figure}[!ht]
\includegraphics[width=0.999\textwidth,angle=0]{figures/plot3VvsZ_giantsD.png} 
\caption{Similar to \autoref{fig:3v}, except that here about five times larger distances are probed with 4.3 million red giant stars
  that have [Fe/H]$>-0.5$ and cylindrical galacto-centric radius between 7 kpc and 10 kpc. The meaning of symbols and lines
  is the same as in \autoref{fig:3v}, and exactly the same models are shown.} 
\label{fig:3vRGs} 
\end{figure}



\subsection{Comparison of predicted and observed shapes of the rotational velocity distribution for disk stars\label{sec:vshape}}

As already suggested in \cite{2008ApJ...684..287I} and confirmed by Bond et al. (see their Figure 6), the shape of the rotational velocity
distribution for disk stars is strongly non-Gaussian (skewed) and it evolves with the distance from the Galactic plane. Histograms based
on Gaia's data shown in \autoref{fig:vPhiShape} confirm this expectation. Furthermore, new data still admit detailed quantitative
modeling of observed distributions as a sum of two Gaussians, with a fixed normalization ratio and a fixed offset of their mean values (see
\autoref{eq:pDvPhi}). Since Gaia's velocity measurement errors are negligible in this context (compared to the much larger width of the observed
velocity distributions; for validation using quasar data, see Appendix), we have reoptimized fits by allowing six parameters to vary. Their best-fit values remain close to the SDSS
values: $f_1=0.40$, $f_2=0.60$, $v_0 = -186 {\rm \, km~s^{-1}}$, $\Delta v_n = 38 {\rm \, km~s^{-1}}$, $a_1 = 22 {\rm \, km~s^{-1}}$ and $a_2= 17
{\rm \, km~s^{-1}}$.  

When the resulting re-optimized pdf, $p(x=v_\phi|Z)$, is used to evaluate the mean $\langle v_\phi \rangle$ and its dispersion
$\sigma_{\phi}$  as functions of $Z$, there is no discernible change for the former compared to \autoref{eq:Dlag}, while the latter is
about 20\% smaller than values given by \autoref{eq:Ddisp} (for illustration of the difference, see the two left panels in \autoref{fig:3v}).
Since there is no implied physics in this two-component statistical model of the skewed velocity distribution, minor numerical adjustments
of the few free model parameters are probably inconsequential. \\

These conclusions based on the FGKM sample are supported by the behavior of the red giant sample, as illustrated in \autoref{fig:vPhiShapeRG}. 
In particular, the vertical gradient of rotational velocity is evident. 

\begin{figure}[!ht]
\includegraphics[width=0.999\textwidth,angle=0]{figures/plotVphiDistr4.png}
\caption{A comparison of the observed rotational velocity distribution ($v_\phi$) for FGKM stars, shown as histograms, and a two-component model (dashed
lines: individual components; solid line: their sum) given by \autoref{eq:pDvPhi} (using updated parameters discussed in text; the same
$Z$-dependent model is shown in all four panels). The four panels show results for two $|Z|$ bins (top row vs. bottom row) and towards North and South
($|b|>80^\circ$, left vs. right). The sample sizes are about 4,000 stars for the nearer $|Z|$ bin and about 900 stars for the more distant bin.} 
\label{fig:vPhiShape} 
\end{figure}


\begin{figure}[!ht]
\includegraphics[width=0.999\textwidth,angle=0]{figures/plotVphiDistr4rg.png}
\caption{Analogous to \autoref{fig:vPhiShape}, except for a subsample of the red giant sample shown in \autoref{fig:3vRGs} restricted to $|b|>60^\circ$.
  The top two panels show the same $|Z|$ bins as in \autoref{fig:vPhiShape}, while the two bottom panels extend to larger $|Z|$. 
  The sample sizes vary from 10,000 stars to 2,000 stars for the most distant bin.} 
\label{fig:vPhiShapeRG} 
\end{figure}



\subsection{Comparison of predicted and observed radial velocity and proper motion sky distributions for disk stars\label{sec:allsky}}


\begin{figure}[!ht]
\includegraphics[width=0.5\textwidth,angle=0]{figures/projview_pmLongData.png}
\includegraphics[width=0.5\textwidth,angle=0]{figures/projview_pmLongDMresiduals_Dslice.png}
\includegraphics[width=0.5\textwidth,angle=0]{figures/projview_pmLatData.png}
\includegraphics[width=0.5\textwidth,angle=0]{figures/projview_pmLatDMresiduals_Dslice.png}
\includegraphics[width=0.5\textwidth,angle=0]{figures/projview_radVelData.png}
\includegraphics[width=0.5\textwidth,angle=0]{figures/projview_radVelDMresiduals_DsliceRV.png}
\caption{The top two left panels show observed distributions of mean proper motion per pixel (top: longitude; middle: latitude)
for 868,859 stars in the 0.4--0.6 kpc distance bin. The bottom left panel shows the observed distribution of mean radial velocity
for a subsample of 400,782 stars with good measurements. The healpix nside=32 maps are shown in the Hammer
projection of galactic coordinates. The maps on the right show residuals (using the same color-coding scale) after
subtracting model values computed using eq.~\ref{eq:Dlag} for $v_\phi$ and with $\langle v_R \rangle = \langle v_Z \rangle = 0$.} 
\label{fig:skyComparison} 
\end{figure}


As the final test, we extrapolate the Bond et al. model from the quarter of the sky observed by SDSS and compare its predictions for
the variation of mean proper motions and radial velocities to Gaia's measurements across the whole sky. The three left panels in
\autoref{fig:skyComparison} show mean proper motion components per pixel and mean radial velocity per pixel for stars with
distances in the range 400 pc to 600 pc. Their strong variation across the sky is evident and is mostly due to projection effects of the
solar motion (and a little bit due to spatial variation of the rotational velocity component with $|Z|$). The panels on the right
show residuals after subtracting proper motion and radial velocity values predicted by the Bond et al. model described in \S~\ref{sec:BondModels}.
As evident, the model fully captures the observed behavior. The only potentially significant feature is observed for radial velocity residuals towards the Galactic
center, at about $0 < b < 30^\circ$. This region was not observed by SDSS and we are not aware of any kinematic or other stellar features
reported for that region (e.g., the famous Sgr dwarf galaxy is located at negative galactic latitudes). A possibility that these residuals
reflect faulty measurements could be, at least in principle, easily checked because most of these stars are relatively bright ($r\sim14$). 

We note that the data maps for stars with distances in the range of 800 pc to 1.2 kpc look qualitatively the same, except that the magnitude
of proper motion is smaller by about a factor of two, as expected. The maps of residuals appear the same, except for the radial velocity
residual ``feature'', which is observed at about 15 degrees higher latitudes than for the first distance bin.

The red giant sample can be used to extend this comparison to distances where the rotational velocity component is much
smaller than locally. We first established that the behavior for red giants in the 0.4--0.6 kpc distance bin is essentially identical
as shown for the FGKM sample in \autoref{fig:skyComparison}. The corresponding plot for red giants in the 2.8--3.2 kpc distance bin 
is shown in \autoref{fig:skyComparisonRG}. Due to about six times larger distances,  the morphology of data panels is rather different
from that in \autoref{fig:skyComparison} (especially for radial velocity, where projection effects of rotational velocity component
dominate). Nevertheless, the model for disk kinematics described in
\S~\ref{sec:BondModels} fully captures the data behavior, as shown by vanishing maps of (data-model) residuals. 

 
\begin{figure}[!ht]
\includegraphics[width=0.5\textwidth,angle=0]{figures/projview_pmLongDataRG.png}
\includegraphics[width=0.5\textwidth,angle=0]{figures/projview_pmLongDMresiduals_DsliceRV_RGdisk.png}
\includegraphics[width=0.5\textwidth,angle=0]{figures/projview_pmLatDataRG.png}
\includegraphics[width=0.5\textwidth,angle=0]{figures/projview_pmLatDMresiduals_DsliceRV_RGdisk.png}
\includegraphics[width=0.5\textwidth,angle=0]{figures/projview_radVelDataRG.png}
\includegraphics[width=0.5\textwidth,angle=0]{figures/projview_radVelDMresiduals_DsliceRV_RGdisk.png}
\caption{Analogous to \autoref{fig:skyComparison}, except for a sample of 1.05 million red giants with [Fe/H]$>-0.5$
  in the 2.8--3.2 kpc distance bin. Note that observed proper motions are much smaller due to about six times larger distances,
  while radial velocity variation with galactic longitude is much stronger due to the projection effects of the rotational velocity component.} 
\label{fig:skyComparisonRG} 
\end{figure}


 
\subsection{Comparison of predicted and observed 3-dimensional velocity distribution for halo stars}

\begin{figure}[ht]
    \centering
    \includegraphics[width=0.999\textwidth,angle=0]{figures/plot3VvsZ_giantsH.png}
    \caption{The variation of spherical velocity components with distance from the plane, $|Z|$, for $\sim$34,000 candidate halo red giants from the solar cylinder with measured radial velocities and [Fe/H]$<-1.2$. Triangles show the mean values in bins of $|Z|$ and the thick dashed lines show the $\pm2\sigma$ envelope around means, where $\sigma$ is standard deviation for each bin (i.e., velocity dispersion). The dot-dashed lines represent the halo velocity ellipsoid model described in Section~\ref{sec:BondModelsHalo}. \label{fig:3vRGhalo}}
\end{figure}
 

\begin{figure}[!ht]
\includegraphics[width=0.999\textwidth,angle=0]{figures/plotVvsV2n.png}
\vskip -2in   
\includegraphics[width=0.999\textwidth,angle=0]{figures/plotVvsV2s.png}
\vskip -2in   
\caption{Illustration of the change of orientation of velocity ellipsoid in cylindrical coordinates for halo stars,
  selected to have [Fe/H]$<-1.2$ and selected from a narrow range of cylindrical coordinates, as shown above
  each panel. The number of stars is about 1,600 for bins with smaller $R$ (left) and 800 for bins on the right.
  The top row shows bins above the plane and the bottom row their symmetric counterparts below the plane
  (the Sun's position is in the middle of the figure). 
  The dashed lines mark the direction towards the Galactic center. The change of the velocity ellipsoid tilt
  in cylindrical coordinates is evident; however, the velocity ellipsoid in spherical coordinates is invariant
  throughout the probed volume, as illustrated by these four spatial bins.} 
\label{fig:haloTilt} 
\end{figure}



Bond et al. found that the kinematics of halo stars can be modeled with a triaxial velocity ellipsoid that is invariant in
spherical coordinates. Motivated by this finding, we show in
% \autoref{fig:3vRGhalo}   % I couldnt make this ref work
Figure 11 
three spherical velocity components for halo red giants as functions of distance from the Galactic plane. Unlike the strong dependence on the velocity dispersion
on $|Z|$ for disk stars seen in Figures~\ref{fig:3v} and \ref{fig:3vRGs}, velocity dispersion for halo stars as measured by
Gaia is spatially invariant, confirming earlier results by Bond et al. We note that small deviation of mean rotational
velocity from zero at $|Z|<1$ kpc seen in the left panel is probably due to sample contamination by much more numerous
disk stars (at $|Z|=0$, halo stars contribute only about 0.5\% of the total count, see Table 10 in \citealt{2008ApJ...673..864J}).  

Gaia data analyzed here provide strong support for a halo velocity ellipsoid that is invariant in spherical coordinates.
When the velocity ellipsoid is expressed in cylindrical coordinates instead, a strong covariance is seen between the $v_Z$ and $v_R$
components. We illustrate this covariance in
% \autoref{fig:haloTilt}.  %I couldnt make this ref work
Figure 12. 
The observed tilt varies with position such that the
velocity ellipsoid points towards the Galactic center (see also \citealt{2019MNRAS.489..910E}). 
 