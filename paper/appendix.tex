\leftline{\bf Validation of proper motion systematic and random uncertainties using quasars}
\vskip 0.1in

We tested Gaia's proper motions and their uncertainties using
spectroscopically confirmed quasars from SDSS Data Release 7. There
are $\sim$367,000 SDSS quasars with Gaia's non-negative proper motion errors.
% (note that columns GAIA\_PM\_RA\_ERR and GAIA\_PM\_DEC\_ERR actually list inverse variance).
Their median  proper motion per
coordinate is about 0.01 mas/yr (indicating no substantial systematic
measurement errors) and the median proper motion magnitude
is about 1.1 mas/yr (indicating typical measurement uncertainty; the
median magnitude of this sample is $G\sim$20; for the FGKM sample
analyzed here, with $G<18$, the proper motion uncertainties are
$<0.15$ mas/yr). 

We have verified that the width of  proper motion per
coordinate normalized by reported uncertainties (i.e. the width of corresponding
$\chi$ distributions) is 1.07 and 1.09, demonstrating Gaia's reliable
estimates of measurement uncertainties.

We didn't find any significant variation of the median quasar proper motion per
coordinate with position on the sky. The only ``interesting feature"
in the data is increased scatter of proper motion per
coordinate measurements in the so-called SDSS Stripe 82 region by
about 50\% compared to the rest of SDSS sky. This effect is easily understood as due to deeper quasar
sample in that region (due to details in SDSS spectroscopic target
selection) and the increase of Gaia's measurement uncertainties with
magnitude (and verified through no substantial increase in the
corresponding $\chi$ distributions). 






